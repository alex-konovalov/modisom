
%%%%%%%%%%%%%%%%%%%%%%%%%%%%%%%%%%%%%%%%%%%%%%%%%%%%%%%%%%%%%%%%%%%%%%%%%%%%%
\Chapter{Nilpotent Quotients}

This chapter contains a description of the nilpotent quotient algorithm
for associative finitely presented algebras. We refer to \cite{Eic11} for 
background on the algorithms used in this Chapter.

%%%%%%%%%%%%%%%%%%%%%%%%%%%%%%%%%%%%%%%%%%%%%%%%%%%%%%%%%%%%%%%%%%%%%%%%%%%%%
\Section{Computing nilpotent quotients}

Let $A$ be a finitely presented algebra in the GAP sense. The following
function can be used to determine the class-$c$ nilpotent quotient of $A$.
The quotient is described by a nilpotent table.

\> NilpotentQuotientOfFpAlgebra( A, c ) F

The output of this function is a nilpotent table with some additional
entries. In particular, there is the additional entry $img$ which 
describes the images of the generators of $A$ in the nilpotent table.

%%%%%%%%%%%%%%%%%%%%%%%%%%%%%%%%%%%%%%%%%%%%%%%%%%%%%%%%%%%%%%%%%%%%%%%%%%%%%
\Section{Example}

\beginexample
gap> F := FreeAssociativeAlgebra(GF(2), 2);;
gap> g := GeneratorsOfAlgebra(F);;
gap> r := [g[1]^2, g[2]^2];;
gap> A := F/r;;
gap> NilpotentQuotientOfFpAlgebra(A,3);
rec( def := [ 1, 2 ], dim := 6, fld := GF(2), 
  img := [ <a GF2 vector of length 6>, <a GF2 vector of length 6> ], 
  mat := [ [  ], [  ] ], rnk := 2, 
  tab := [ [ <a GF2 vector of length 6>, <a GF2 vector of length 6>, 
          [ 0*Z(2), 0*Z(2), 0*Z(2), 0*Z(2), Z(2)^0, 0*Z(2) ], 
          [ 0*Z(2), 0*Z(2), 0*Z(2), 0*Z(2), 0*Z(2), 0*Z(2) ] ], 
      [ <a GF2 vector of length 6>, <a GF2 vector of length 6>, 
          [ 0*Z(2), 0*Z(2), 0*Z(2), 0*Z(2), 0*Z(2), 0*Z(2) ], 
          [ 0*Z(2), 0*Z(2), 0*Z(2), 0*Z(2), 0*Z(2), Z(2)^0 ] ] ], 
  wds := [ ,, [ 2, 1 ], [ 1, 2 ], [ 1, 3 ], [ 2, 4 ] ], 
  wgs := [ 1, 1, 2, 2, 3, 3 ] )
\endexample

%%%%%%%%%%%%%%%%%%%%%%%%%%%%%%%%%%%%%%%%%%%%%%%%%%%%%%%%%%%%%%%%%%%%%%%%%%%%%
\Chapter{Relatively free Algebras}

As described in \cite{Eic11}, the nilpotent quotient algorithm also allows
to determine certain relatively free algebras; that is, algebras that are
free within a variety.

%%%%%%%%%%%%%%%%%%%%%%%%%%%%%%%%%%%%%%%%%%%%%%%%%%%%%%%%%%%%%%%%%%%%%%%%%%%%%
\Section{Computing Kurosh Algebras}

\> KuroshAlgebra( d, n, F ) F

determines a nilpotent table for the largest associative algebra on
$d$ generators over the field $F$ so that every element $a$ of the 
algebra satisfies $a^n = 0$.

\> ExpandExponentLaw( T, n )

suppose that $T$ is the nilpotent table of a Kurosh algebra of exponent
$n$ defined over a prime field. This function determines polynomials 
describing the corresponding Kurosh algebras over all fields with the same 
characteristic as the prime field.

%%%%%%%%%%%%%%%%%%%%%%%%%%%%%%%%%%%%%%%%%%%%%%%%%%%%%%%%%%%%%%%%%%%%%%%%%%%%%
\Section{A Library of Kurosh Algebras}

The package contains a library of Kurosh algebras. This can be accessed
as follows.

\> KuroshAlgebraByLib(d, n, F) F

At current, the library contains the Kurosh algebras for 
$n=2$, 
$(d,n) = (2,3)$, 
$(d,n) = (3,3)$ and $F = \Q$ or $|F| \in \{2,3,4\}$,
$(d,n) = (4,3)$ and $F = \Q$ or $|F| \in \{2,3,4\}$,
$(d,n) = (2,4)$ and $F = \Q$ or $|F| \in \{2,3,4,9\}$,
$(d,n) = (2,5)$ and $F = \Q$ or $|F| \in \{2,3,4,5,8,9\}$.

%%%%%%%%%%%%%%%%%%%%%%%%%%%%%%%%%%%%%%%%%%%%%%%%%%%%%%%%%%%%%%%%%%%%%%%%%%%%%
\Section{Example}

\beginexample
gap> KuroshAlgebra(2,2,Rationals);
... some printout ..
rec( bas := [ [ 1, 0, 0, 0 ], [ 0, 1, 1, 0 ], [ 0, 0, 0, 1 ], [ 0, 1, 0, 0 ] ]
    , com := false, dim := 3, fld := Rationals, rnk := 2, 
  tab := [ [ [ 0, 0, 0 ], [ 0, 0, -1 ] ], [ [ 0, 0, 1 ], [ 0, 0, 0 ] ] ], 
  wds := [ ,, [ 2, 1 ] ], wgs := [ 1, 1, 2 ] )
\endexample



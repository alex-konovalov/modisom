
%%%%%%%%%%%%%%%%%%%%%%%%%%%%%%%%%%%%%%%%%%%%%%%%%%%%%%%%%%%%%%%%%%%%%%%%%%%%%
\Chapter{Automorphism groups and Canonical Forms}

We refer to \cite{Eic07} for background on the algorithms used in this 
Chapter. Throughout the chapter, we assume that $F$ is a finite field.

%%%%%%%%%%%%%%%%%%%%%%%%%%%%%%%%%%%%%%%%%%%%%%%%%%%%%%%%%%%%%%%%%%%%%%%%%%%%%
\Section{Automorphism groups}

Let $T$ be a nilpotent table over $F$. The following function can be used 
to determine the automorphism group of the algebra described by $T$. The
automorphism group is determined as subgroup of $GL(T.dim, T.fld)$ given 
by generators and its order. There is a variation available to determine
the automorphism group of a modular group algebra $FG$, where $F$ is a finite
field and $G$ is a $p$-group.

\> AutGroupOfTable( T ) F 
\> AutGroupOfRad( FG ) F 

In both cases, the automorphism group is described by a record. The
matrices in the lists $glAutos$ and $agAutos$ generate together the 
automorphism group. The matrices in $agAutos$ generate a $p$-group.
The entry $size$ contains the order of the automorphism group.

%%%%%%%%%%%%%%%%%%%%%%%%%%%%%%%%%%%%%%%%%%%%%%%%%%%%%%%%%%%%%%%%%%%%%%%%%%%%%
\Section{Canonical forms}

Let $T$ be a nilpotent table. The following function can be used to determine
the automorphism group of $T$ if the underlying field of $T$ is finite. The
canonical form is a nilpotent table which is unique for the isomorphism type
of the algebra defined by $T$. Again there a variation available for modular
group algebras. 

\> CanonicalFormOfTable( T ) F 
\> CanonicalFormOfRad( FG ) F 

The automorphism group of $T$ is determined as a side-product of computing
the canonical form. The following functions can be used to return both.

\> CanoFormWithAutGroupOfTable( T ) F
\> CanoFormWithAutGroupOfRad( FG ) F

In both cases, these functions return a record with entries $cano$ and
$auto$.

%%%%%%%%%%%%%%%%%%%%%%%%%%%%%%%%%%%%%%%%%%%%%%%%%%%%%%%%%%%%%%%%%%%%%%%%%%%%%
\Section{Examples}

We compute the automorphism group and a canonical form for the 
modular group algebra of the dihedral group of order 8.

\beginexample
gap> A := GroupRing(GF(2), SmallGroup(8,3));;
gap> T := TableByWeightedBasisOfRad(A);;
gap> C := CanoFormWithAutGroupOfTable(T);;

# check that the canonical form is not equal to T
gap> CompareTables(C.cano, T);
false

# the order of the automorphism group
gap> C.auto.size;
512

# the entries of the canonical table as far as they are bounded
gap> C.cano.tab;
[ [ <a GF2 vector of length 7>, <a GF2 vector of length 7>, 
      [ 0*Z(2), 0*Z(2), 0*Z(2), 0*Z(2), Z(2)^0, 0*Z(2), 0*Z(2) ], 
      [ 0*Z(2), 0*Z(2), 0*Z(2), 0*Z(2), 0*Z(2), 0*Z(2), 0*Z(2) ], 
      [ 0*Z(2), 0*Z(2), 0*Z(2), 0*Z(2), 0*Z(2), 0*Z(2), 0*Z(2) ], 
      [ 0*Z(2), 0*Z(2), 0*Z(2), 0*Z(2), 0*Z(2), 0*Z(2), Z(2)^0 ] ], 
  [ <a GF2 vector of length 7>, <a GF2 vector of length 7>, 
      [ 0*Z(2), 0*Z(2), 0*Z(2), 0*Z(2), 0*Z(2), 0*Z(2), 0*Z(2) ], 
      [ 0*Z(2), 0*Z(2), 0*Z(2), 0*Z(2), 0*Z(2), Z(2)^0, 0*Z(2) ], 
      [ 0*Z(2), 0*Z(2), 0*Z(2), 0*Z(2), 0*Z(2), 0*Z(2), Z(2)^0 ], 
      [ 0*Z(2), 0*Z(2), 0*Z(2), 0*Z(2), 0*Z(2), 0*Z(2), 0*Z(2) ] ], 
  [ [ 0*Z(2), 0*Z(2), 0*Z(2), 0*Z(2), 0*Z(2), 0*Z(2), 0*Z(2) ], 
      [ 0*Z(2), 0*Z(2), 0*Z(2), 0*Z(2), 0*Z(2), Z(2)^0, 0*Z(2) ], 
      [ 0*Z(2), 0*Z(2), 0*Z(2), 0*Z(2), 0*Z(2), 0*Z(2), Z(2)^0 ], 
      [ 0*Z(2), 0*Z(2), 0*Z(2), 0*Z(2), 0*Z(2), 0*Z(2), 0*Z(2) ] ], 
  [ [ 0*Z(2), 0*Z(2), 0*Z(2), 0*Z(2), Z(2)^0, 0*Z(2), 0*Z(2) ], 
      [ 0*Z(2), 0*Z(2), 0*Z(2), 0*Z(2), 0*Z(2), 0*Z(2), 0*Z(2) ], 
      [ 0*Z(2), 0*Z(2), 0*Z(2), 0*Z(2), 0*Z(2), 0*Z(2), 0*Z(2) ], 
      [ 0*Z(2), 0*Z(2), 0*Z(2), 0*Z(2), 0*Z(2), 0*Z(2), Z(2)^0 ] ], 
  [ [ 0*Z(2), 0*Z(2), 0*Z(2), 0*Z(2), 0*Z(2), 0*Z(2), 0*Z(2) ], 
      [ 0*Z(2), 0*Z(2), 0*Z(2), 0*Z(2), 0*Z(2), 0*Z(2), Z(2)^0 ] ], 
  [ [ 0*Z(2), 0*Z(2), 0*Z(2), 0*Z(2), 0*Z(2), 0*Z(2), Z(2)^0 ], 
      [ 0*Z(2), 0*Z(2), 0*Z(2), 0*Z(2), 0*Z(2), 0*Z(2), 0*Z(2) ] ] ]
\endexample

%%%%%%%%%%%%%%%%%%%%%%%%%%%%%%%%%%%%%%%%%%%%%%%%%%%%%%%%%%%%%%%%%%%%%%%%%%%%%
\Chapter{The modular isomorphism problem}

An application of the methods in this package has been the checking
of the modular isomorphism problems for the groups of order dividing
$2^8$, $3^6$ and $2^9$ \cite{Eic07,EKo10}. This section contains the 
functions used for this purpose. 

%%%%%%%%%%%%%%%%%%%%%%%%%%%%%%%%%%%%%%%%%%%%%%%%%%%%%%%%%%%%%%%%%%%%%%%%%%%%%
\Section{Computing and checking bins}

\> BinsByGT( p, n ) F

returns a partion of the list $[1..NumberSmallGroups(p^n)]$ into 
sublists so that the modular group algebras of two groups 
SmallGroup($p^n$, $i$) and SmallGroup($p^n$, $j$) can not be
isomorphic if $i$ and $j$ are in different lists. The function
BinsByGT uses various group theoretic invariants to split the
groups of order $p^n$ in bins.

\> CheckBin( p, n, k, bin ) F

For $i \in bin$ let $G_i$ denote SmallGroup($p^n$, $i$) and let $A_i$
be the augementation ideal of $F G_i$. This function computes and
compares the canonical forms of the algebras $A_i / A_i^j$ for every
$i \in bin$ and increasing $j \in \{1, \ldots, k+1\}$. 

At each level $j$ it splits the current bins into sub-bins according 
to the different canonical forms of $A_i/A_i^j$. Bins of length 1 are 
then discarded.

The function returns if no further bins are available or if $j=k+1$ is
reached. In the later case the function returns the remaining bins. 

%%%%%%%%%%%%%%%%%%%%%%%%%%%%%%%%%%%%%%%%%%%%%%%%%%%%%%%%%%%%%%%%%%%%%%%%%%%%%
\Section{Examples}

We show how to check the modular isomorphism problem for the groups
of order 64. We first use BinsByGT to determine bins and we then check
the first of the resulting bins with CheckBin. The fact that CheckBin
ends with an empty list of bins shows that all groups are splitted.

\beginexample
gap> bins := BinsByGT(2,6);
refine by abelian invariants of group (Sehgal/Ward) 
13 bins with 256 groups 
refine by abelian invariants of center (Sehgal/Ward) 
30 bins with 237 groups 
refine by lower central series (Sandling) 
32 bins with 127 groups 
refine by jennings series (Passi+Sehgal/Ritter+Sehgal) 
36 bins with 123 groups 
refine by conjugacy classes (Roggenkamp/Wursthorn) 
16 bins with 36 groups 
refine by elem-ab subgroups (Quillen) 
  start bin 1 of 16
  start bin 2 of 16
  start bin 3 of 16
  start bin 4 of 16
  start bin 5 of 16
  start bin 6 of 16
  start bin 7 of 16
  start bin 8 of 16
  start bin 9 of 16
  start bin 10 of 16
  start bin 11 of 16
  start bin 12 of 16
  start bin 13 of 16
  start bin 14 of 16
  start bin 15 of 16
  start bin 16 of 16
9 bins with 21 groups 
[ [ 13, 14 ], [ 18, 19 ], [ 20, 22 ], [ 97, 101 ], [ 108, 110 ], 
  [ 155, 157, 159 ], [ 156, 158, 160 ], [ 173, 176 ], [ 179, 180, 181 ] ]

gap> CheckBin(2,6,bins[1]);
compute tables through power series 
  determined table for 1
  determined table for 2

refine bin 
  weights yields bins [ [ 1, 2 ] ]
  layer 1 yields bins [ [ 1, 2 ] ]
  layer 2 yields bins [ [ 1, 2 ] ]
  layer 3 yields bins [ [ 1, 2 ] ]
  layer 4 yields bins [  ]
\endexample

